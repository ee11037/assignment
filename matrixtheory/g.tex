\let\negmedspace\undefined
\let\negthickspace\undefined
\documentclass[journal,12pt,twocolumn]{IEEEtran}
\usepackage{cite}
\usepackage{amsmath,amssymb,amsfonts,amsthm}
\usepackage{algorithmic}
\usepackage{graphicx}
\usepackage{textcomp}
\usepackage{xcolor}
\usepackage{txfonts}
\usepackage{listings}
\usepackage{enumitem}
\usepackage{mathtools}
\usepackage{gensymb}
\usepackage{comment}
\usepackage[breaklinks=true]{hyperref}
\usepackage{tkz-euclide} 
\usepackage{listings}
\usepackage{gvv}  
\usepackage{tikz}
\usepackage{circuitikz} 
\usepackage{caption}
\def\inputGnumericTable{}              
\usepackage[latin1]{inputenc}          
\usepackage{color}                    
\usepackage{array}                     
\usepackage{longtable}                 
\usepackage{calc}                     \usepackage{multirow}                  
\usepackage{hhline}                    
\usepackage{ifthen}                    
\usepackage{lscape}
\usepackage{amsmath}
\newtheorem{theorem}{Theorem}[section]
\newtheorem{problem}{Problem}
\newtheorem{proposition}{Proposition}[section]
\newtheorem{lemma}{Lemma}[section]
\newtheorem{corollary}[theorem]{Corollary}
\newtheorem{example}{Example}[section]
\newtheorem{definition}[problem]{Definition}
\newcommand{\BEQA}{\begin{eqnarray}}
\newcommand{\EEQA}{\end{eqnarray}}
\newcommand{\define}{\stackrel{\triangle}{=}}
\theoremstyle{remark}
\newtheorem{rem}{Remark}

%\bibliographystyle{ieeetr}
\begin{document}
%

\bibliographystyle{IEEEtran}




\title{
%	\logo{
13. Properties of Triangle


%	}
}
\author{Manognya Kundarapu

(EE24BTECH11037)\\
\vspace{0.5cm}
Section-A JEE Advanced/IIT-JEE
}	





\maketitle

\newpage
\bigskip

\renewcommand{\thefigure}{\theenumi}
\renewcommand{\thetable}{\theenumi}



    
    




{E. Subjective Problems}

\begin{enumerate}
    \item A triangle $ABC$ has sides $AB=AC=5cm$ and $BC=6cm$ Triangle $A'B'C'$ is the reflection of the triangle $ABC$ in a line parallel to $AB$ is placed at a distance $2cm$ from $AB$, outside the triangle $ABC$. Triangle $A''B''C''$ is the reflection of the triangle $A'B'C$ in a line parallel to $B'C'$ placed at a distance of $2cm$ from $B'C'$ placed at a distance of $2cm$ from $B'C'$ outside the triangle $A'B'C'$. Find the distance between $A$ and $A''$.\hfill(1978)
    \item (a) If a circle is inscribed in aright angled triangle $ABC$ with the right angle at $B$, show that the diameter of the circle is equal to $AB+BC-AC$.\\
    (b) If a triangle is inscribed in a circle, then the product of any two sides of the triangle is equal to the product of the diameter and the perpendicular distance of the third side from the opposite vertex. Prove the above statement.
    
    \hfill(1979)
    \item (a) A ballon is observed simultaneously from three points $A,B$ and $C$ on a straight road directly beneath it. The angular elevation at $B$ is twice that at $A$ and the angular elevation at $C$ is thrice that at $A$. If the distance between $A$ and $B$ is $a$ and the distance between $B$ and $C$ is $b$, find the height of the balloon in terms of $a$ and $b$.\\
    (b) Find the area of the smaller part of a disc of radius $10cm$, cut off by a chord $AB$ which subtends an angle of $22\frac{1}{2}$\textdegree at the circumference.

    \hfill(1979)
    \item $ABC$ is a triangle. $D$ is the middle point of $BC$. If $AD$ is perpendicular to $AC$,then prove that \\
    $\cos A \cos C = \frac{2(c^2 - a^2)}{3ac}$\hfill(1980)
    \item If in a triangle $ABC$,\\ $\cos A \cos B+\sin A \sin B \sin C=1$, Show that $a:b:c=1:1:\sqrt{2}$\hfill(1986-5 Marks)
    \item A sign-post in the form of an isosceles triangle $ABC$ is mounted on a pole of height $h$ fixed to the ground. The base $BC$ of the triangle is parallel to the ground. A man standing on the ground at a distance $d$ from the sign-post finds that the top vertex $A$ of the triangle subtends an angle $\beta$ and either of the other two vertices subtends the same angle $\alpha$ at his feet. Find the area of the triangle.\hfill(1988-5 Marks)
    \item $ABC$ is a triangular park with $AB=aAC=100m$. A television tower stands at the midpoint of $BC$. The angles of elevation of the top of the tower at $A,B,C$ are $45$\textdegree, $60$\textdegree, $60$\textdegree, respectively. Find the height of the tower.\hfill(1989-5 Marks)
    \item Let the angles $A,B,C$ of a triangle $ABC$ be in A.P.and let $b:c=\sqrt{3}:\sqrt{2}$.Find the angle $A$.
    
    \hfill(1981-2 Marks)
    \item A vertical pole stands at a point $Q$ on a horizontal ground. $A$ and $B$ are points on the ground, $d$ metres apart. The pole subtends angles $\alpha$ and $\beta$ at $A$ and $B$ respectively. $AB$ subtends an angle $\gamma$ at $Q$. Find the height of the pole.\hfill(1982-3 Marks)
    \item The ex-radii $r_1,r_2,r_3$ of $\triangle ABC$ are in H.P. Show that its sides $a,b,c$ are in A.P.

    \hfill(1983-3 Marks)
    \item For a triangle $ABC$ it is given that $\cos A+\cos B+\cos C=\frac{3}{2}$. Prove that the triangle is equilateral.\hfill(1984-4 Marks)
    \item With usual notation, if in a triangle $ABC$;\\
    $\frac{b+c}{11}=\frac{c+a}{12}=\frac{a+b}{13}$ then prove that\\
    $\frac{\cos A}{7}=\frac{\cos B}{19}=\frac{\cos C}{25}$.\hfill(1984-4 Marks)
    \item A ladder rests against a wall at an angle $\alpha$ to the horizontal. Its foot is pulled away from the wall through a distance $a$, so that it slides $a$ distance $b$ down the wall making an angle $\beta$ with the horizontal,\\
    Show that $a=\tan \frac{1}{2}(\alpha+\beta)$\hfill(1985-5 Marks)
    \item $ABC$ is a triangle with $AB=AC$. $D$ is any point on the side $BC. E$ and $F$ are points on the side $AB$ and $AC$, respectively, such that $DE$ is parallel to $AC$, and $DF$ is parallel to $AB$. Prove that\\
    $DF+FA+AE+ED=AB+AC$

    \hfill(1980)
    \item In a triangle $ABC$, the median to the side $BC$ is of length $\frac{1}{\sqrt{11-6\sqrt{3}}}$ and it divides the angle $A$ into angles $30$\textdegree and $45$\textdegree.Find the length of the side $BC$.\hfill(1985-5 Marks)
\end{enumerate} 
\end{document}
